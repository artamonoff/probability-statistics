% !TEX root = prob-exercises.tex

\section{Непрерывные распределения}

\subsection{Плотность, функция распределения, математическое ожидание, дисперсия}

\begin{exercise}
Пусть случайная величина \(X\) имеет плотность
\[
	f(x)=\begin{cases}
	cx, & x\in[0,1] \\ 0, & \text{иначе}
	\end{cases}
\]
\begin{enumerate}
	\item Найдите нормировочный множитель \(c\) и нарисуйте график плотности
	\item Вычислите вероятности
	\begin{align*}
		\Prob&(X>0.5) & \Prob&(0.25<X<0.75) & \Prob(-1<X<0.5)
	\end{align*}
	\item Вычислите \(\Exp(X)\) и \(\Var(X)\)
	\item Найдите функцию распределения \(F(x)\) и нарисуйте её график
\end{enumerate}
\end{exercise}

\begin{exercise}
Пусть случайная величина \(X\) имеет плотность 
\[
	f(x)=\begin{cases}
	cx^{\lambda-1}, & x\in[0,1] \\ 0, & \text{иначе}
	\end{cases}
\]
(\(\lambda\)>0 -- параметр распределения)
\begin{enumerate}
	\item Найдите нормировочный множитель \(c\) и нарисуйте график плотности \(f\)
	\item Вычислите вероятности
	\begin{align*}
		\Prob&(X>0.5) & \Prob&(0.25<X<0.75) & \Prob(-1<X<0.5)
	\end{align*}
	\item Вычислите \(\Exp(X)\) и \(\Var(X)\)
	\item Найдите функцию распределения \(F\) и нарисуйте её график
\end{enumerate}
\textit{Замечание}: графики \(f\) и  \(F\) нарисуйте при \(0<\lambda<1\) и при \(\lambda\geq 1\)
\end{exercise}

\begin{exercise}
Пусть случайная величина \(X\) имеет плотность
\[
	f(x)=\begin{cases}
	cx(1-x), & x\in[0,1] \\ 0, & \text{иначе}
	\end{cases}
\]
\begin{enumerate}
	\item Найдите нормировочный множитель \(c\) и нарисуйте график плотности
	\item Вычислите вероятности
	\begin{align*}
		\Prob&(X<0.5) & \Prob&(0.25<X<0.75) & \Prob(-5<X<0.25)
	\end{align*}
	\item Вычислите \(\Exp(X)\) и \(\Var(X)\)
	\item Найдите функцию распределения \(F(x)\) и нарисуйте её график
\end{enumerate}
\end{exercise}

\subsection{Стандартные распределения}

\begin{exercise}
Для распределения \(\Gauss(0,1)\) вычислите
\begin{align*}
	&\phi(1) & &\phi(2) & &\phi(-0.5) & &\phi(-1.5) & &\Phi(1) & &\Phi(2) & &\Phi(-1) & &\Phi(-2)
\end{align*}
\end{exercise}

\begin{exercise}
Для распределения \(\Gauss(1,0.5^2)\) вычислите значение плотности и функции распределения в точках
\[
	x=\{-3, -2, -1.5, -1, -0.5, 0, 0.5, 1, 1.5, 2, 2.5, 3\}
\]
\end{exercise}

\subsection{Критические значения}

\textbf{Замечание}: все вычисления необходимо сделать в MS Excel/Python

\begin{exercise}
Для уровней значимости: 1\%, 5\%, 10\% вычислите (двусторонние) 
критические значения распределения \(\Gauss(0,1)\)
\end{exercise}

\begin{exercise}
Для уровней значимости: 1\%, 5\%, 10\% вычислите (двусторонние) 
критические значения следующих распределений
\begin{align*}
	&t_{10} & &t_{100} & &t_{250} & &t_{500}
\end{align*}
\end{exercise}

\begin{exercise}
Для уровней значимости: 1\%, 5\%, 10\% вычислите
критические значения следующих распределений
\begin{align*}
	&\chi^2_{2} & &\chi^2_{5} & &\chi^2_{10} & &\chi^2_{20}
\end{align*}
\end{exercise}

\begin{exercise}
Для уровней значимости: 1\%, 5\%, 10\% вычислите
критические значения следующих распределений
\begin{align*}
	&F_{2,100} & &F_{5, 300} & &F_{10, 1000} & &F_{20, 1500}
\end{align*}
\end{exercise}
% !TEX root = prob-exercises.tex

%\setcounter{problem}{0}

\section{Дискретные случайные величины}

\subsection{Одномерные распределения}

\begin{exercise}
В урне содержится 3 белых и 3 черных шара. Случайным образом извлекаются
2 шара. Пусть случайная величина \(X\) -- число белых шаров среди выбранных.
\begin{enumerate}
	\item Найдите таблицу распределения \(X\)
	\item Вычислите \(\Exp(X)\), \(\Var(X)\), \(\sigma(X)\) и моду распределения
	\item Вычислите вероятности
	\begin{align*}
		&\Prob(X<2) & &\Prob(X\geq1) & &\Prob(0<X<3)
	\end{align*}
	\item Нарисуйте график функции распределения \(F\).
\end{enumerate}
\textit{Замечание}: \(X\sim Hypergeom(6,3,2)\)
\end{exercise}

\begin{exercise}
В урне содержится 4 белых и 2 черных шара. Случайным образом извлекаются
3 шара. Пусть случайная величина \(X\) -- число белых шаров среди выбранных.
\begin{enumerate}
	\item Найдите таблицу распределения \(X\)
	\item Вычислите \(\Exp(X)\), \(\Var(X)\), \(\sigma(X)\) и моду распределения
	\item Вычислите вероятности
	\begin{align*}
		&\Prob(X<3) & &\Prob(X>1) & &\Prob(1<X<3)
	\end{align*}
	\item Нарисуйте график функции распределения \(F\).
\end{enumerate}
\textit{Замечание}: \(X\sim Hypergeom(6,4,2)\)
\end{exercise}

\begin{exercise}
В урне содержится 3 белых и 4 черных шара. Случайным образом извлекаются
4 шара. Пусть случайная величина \(X\) -- число белых шаров среди выбранных.
\begin{enumerate}
	\item Найдите таблицу распределения \(X\)
	\item Вычислите \(\Exp(X)\), \(\Var(X)\), \(\sigma(X)\) и моду распределения
	\item Вычислите вероятности
	\begin{align*}
		&\Prob(X<3) & &\Prob(X>0) & &\Prob(0<X<3)
	\end{align*}
	\item Нарисуйте график функции распределения \(F\).
\end{enumerate}
\textit{Замечание}: \(X\sim Hypergeom(7,2,4)\)
\end{exercise}

\subsection{Двумерные распределения}


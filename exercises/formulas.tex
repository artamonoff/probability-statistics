% !TEX root = prob-exercises.tex

\section{Основные формулы}

\subsection{Основы теории вероятностей}

Комбинаторика:
\begin{itemize}
	\item Число перестановок \(P_n=n!\)
	\item Число сочетаний \(C_n^k=\binom{n}{k}=\frac{n!}{k!(n-k)!}\)
	\item Число размещений \(A_n^k=\frac{n!}{(n-k)!}\)
\end{itemize}
<<Классическое>> определение вероятностей (модель равновероятных элементарных исходов)
\[
	\Prob(A)=\frac{\#(A)}{\#(\Omega)}
\]
Формула суммы событий
\[
	\Prob(A+B)=\Prob(A)+\Prob(A)-\Prob(AB)
\]
Условная вероятность \(B\) при условии (наблюдения) \(A\)
\begin{align*}
	\Prob(B|A)&=\frac{\Prob(AB)}{\Prob(A)} & \Prob(A)&\ne0
\end{align*}

Пусть \(\{H_i\}_{i=1}^n\) -- полная группа событий (гипотезы).
Тогда для произвольного события  \(A\)
\begin{itemize}
	\item Формула полной вероятности
	\[
		\Prob(A)=\sum_{i=1}^n \Prob(A|H_i)\Prob(H_i)
	\]
	\item Формула Байеса
	\begin{align*}
		\Prob(H_i|A)&=\frac{\Prob(A|H_i)\Prob(H_i)}{\Prob(A)} &
		i&=1,\ldots,n
	\end{align*}
\end{itemize}

\subsection{Дискретные случайные величины}

\subsection{Непрерывные случайные величины}

\subsubsection{Общие распределения}

Пусть \(X\) -- непрерывной распределённая случайная величина. Функция распределения \(F\) и плотность \(f\)
определяются как
\begin{align*}
	F(x)&=\Prob(X\leq x) & f(x)&=F'(x) & x&\in\R
\end{align*}
Свойства функции распределения:
\begin{align*}
	0&\leq F(x)\leq1 & &F \uparrow & 
	\lim_{x\to-\infty}F(x)&=0 & \lim_{x\to+\infty}F(x)&=1
\end{align*}
Свойства плотности
\begin{align*}
	f(x)&\geq 0 & \int_\R f(t)dt&=1 & F(x)&=\int_{-\infty}^x f(t)dt
\end{align*}
Вычисление вероятностей
\begin{align*}
	\left.\begin{aligned}
	&\Prob(a<X<b) \\ & \Prob(a\leq X\leq b) \\ 
	& \Prob(a<X\leq b) \\ &\Prob(a\leq X< b)
	\end{aligned}
	\right\}&=F(b)-F(a)=\int_a^b f(t)dt\\
	\left.\begin{aligned}
	&\Prob(X<b) \\ & \Prob(X\leq b)
	\end{aligned}
	\right\}&=F(b)=\int_{-\infty}^b f(t)dt\\
	\left.\begin{aligned}
	&\Prob(a<X) \\ & \Prob(a\leq X)
	\end{aligned}
	\right\}&=1-F(a)=\int_{a}^{+\infty} f(t)dt
\end{align*}
Математическое ожидание (\textbf{если интеграл сходится!})
\[
	\Exp(X)=\int_\R xf(x)dx
\]
Если интеграл расходится, то математическое ожидание не существует.

Момент порядка \(k\in\N\)  (\textbf{если интеграл сходится!})
\[
	\Exp(X^k)=\int_\R x^kf(x)dx
\]
Дисперсия (если конечен момент второго порядка!)
\[
	\Var(X)=\Exp(X^2)-(\Exp(X))^2\geq0
\]

\subsubsection{Стандартные распределения}

\paragraph{Нормальное (гауссово) распределение}

Стандартное гауссово или нормальное распределение \(\Gauss(0,1)\):
\begin{itemize}
	\item плотность и функция распределения
	\begin{align*}
		\phi(x)&=\frac{1}{\sqrt{2\pi}}\exp(-x^2/2) &
		\Phi(x)&=\int^x_{-\infty}\phi(t)dt
	\end{align*}
	\item Математическое ожидание и дисперсия
	\begin{align*}
		\Exp(X)&=0 & \Var(X)&=1
	\end{align*}
\end{itemize}
Общее нормальное распределение \(\Gauss(\mu,\sigma^2)\)
\begin{itemize}
	\item плотность и функция распределения
	\begin{align*}
		f(x)&=\frac{1}{\sigma}\phi\left(\frac{x-\mu}{\sigma}\right)=
		\frac{1}{\sqrt{2\pi\sigma^2}}\exp\left(-\frac{(x-\mu)^2}{2\sigma^2}\right) \\
		F(x)&=\Phi\left(\frac{x-\mu}{\sigma}\right)
	\end{align*}
	\item Математическое ожидание и дисперсия
	\begin{align*}
		\Exp(X)&=\mu & \Var(X)&=\sigma^2
	\end{align*}
\end{itemize}

\paragraph{Равномерное распределение} \(U[a,b]\) на отрезке
\begin{itemize}
	\item плотность и функция распределения
	\begin{align*}
		f(x)&=\begin{cases}
			\frac{1}{b-a}, & a\leq x\leq b\\
			0, & \text{иначе}
		\end{cases} \\
		F(x)&=\begin{cases}
			0, & x< a\\
			\frac{x-a}{b-a}, & a\leq x\leq b\\
			1, & x>b
		\end{cases}
	\end{align*}
	\item Математическое ожидание и дисперсия
	\begin{align*}
		\Exp(X)&=\frac{a+b}{2} & \Var(X)&=\frac{(b-a)^2}{12}
	\end{align*}
\end{itemize}

\paragraph{Экспоненциальное распределение} \(Exp(\lambda)\) (параметр \(\lambda>0\))
\begin{itemize}
	\item плотность и функция распределения
	\begin{align*}
		f(x)&=\begin{cases}
			\lambda \exp(-\lambda x), & x\geq0\\
			0, & x<0
		\end{cases} \\
		F(x)&=\begin{cases}
			1-\exp(-\lambda x), & x\geq0 \\
			0, & x<0
		\end{cases}
	\end{align*}
	\item Математическое ожидание и дисперсия
	\begin{align*}
		\Exp(X)&=\frac{1}{\lambda} & \Var(X)&=\frac{1}{\lambda^2}
	\end{align*}
\end{itemize}